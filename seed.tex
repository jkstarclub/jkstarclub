\documentclass{article}
\usepackage[margin=1in]{geometry}
\usepackage[utf8]{inputenc}
\usepackage{algorithm}
\usepackage[noend]{algpseudocode}
\usepackage{amsmath,amssymb,amsfonts,amsthm}
\usepackage[numbers]{natbib}
\usepackage{hyperref}
\usepackage{booktabs}
\usepackage{siunitx}
\usepackage{xcolor}
\newcommand{\stnote}[1]{\textcolor{blue}{\textbf{ST: #1}}}
\DeclareMathOperator{\atantwo}{atan2}
\DeclareMathOperator{\asin}{asin}
\title{The Amped-Up Rotation Theorem}
\author{
  \parbox{0.3\linewidth}{\centering JOHN\\Larboard Canada}
  }
\date{20}
\begin{document}
\maketitle

This article is to celebrate the puzzle of flipping rotation and to unpack the puzzle into simple usage examples to explain the 3D motor. The explanation can be used to solve three problems: flip, noise, and overheat. One flip is the structural immorality. Two noises are the absolute flow. Three overheat show exit.

\section{Overturn}
\url{https://overleaf.com}

\section{Overflow}
\url{https://mathoverflow.net}

\section{Propeller}
\newline{Wayback machine}
\newline{Wayback machine}
\newline{}
\newline{Welcome}
\begin{align}
    \hat{x_t} &= \frac{1}{t} \left[ \hat{x}_{t-1} \times (t-1) + z_t \right].
\end{align}
Thank you
\begin{align}
    \hat{exit} &= \frac{old+new}{2} \left[ \hat{flip}_{turn} \times (transparent) + burn \right].
\end{align}
See you again

\bibliographystyle{plainnat}
\bibliography{main}
\end{document}

